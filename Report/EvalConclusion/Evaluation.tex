\chapter{Evaluation and Conclusion}
\label{chap:Eval}

\section{Evaluation} 
\subsubsection{Evolution of the Team} At the beginning of the course, the project teams of the course were randomly assigned. Each team member comprising the Alternative Spaces project development team comes from different backgrounds, with different cultures, experiences and skills. Initially the most noticable challenge seemed to lie with language, with five Norwegian and two international students. With the international students having too little experience with Norwegian, the main communication of the group had to be in English. Fortunately, all group members are to a satisfying extent experienced with speaking English. 
\subsubsection{Becoming a Team} Upon meeting the customer and being introduced to a vague project without specific tasks detailing what was expected of us, we sat down as a group and began our initial phase of the project. This included regular group meetings where we sat down brainstorming the concept and idea development. Among other things this included deciding whether the prototype should be user oriented, interest oriented or location oriented. The project had a lot of freedom, but this also meant more responsibility for own work. Noticing our differences, the group dynamics seemed weak at this time. The group was unstructured and without a leader, and everything was done together as a group.
\subsubsection{Improving the Work Process} After the idea development had begun converging towards an idea that the group and the customer both seemed to like, the implementation and design phase started. As the group no longer worked together for each work session, the lack of group organization and difference in experience of each member in the group started having a negative effect on the work and workflow. This resulted in unclear tasks, which reduced the work efficiency of the group, and caused a lack in motivation and hours of work.
\paragraph{} To deal with this problem we sat down to identify the skills of each person, so that clear roles could be assigned each member. A project leader and a technical leader was assigned to organize and structure the group as well as keeping track on each member and being mainly responsible for dealing with problems. Additionally, work assignments now had to be more precisely defined so that when members were assigned a task, there was no ambiguity as to what the person should do, the time it should take and when it had to be done. As it turned out meeting every day was impossible, these measures made the entire team more effective and made it possible for every member to finish their work assignments within the given timeframes. Still, we tried to meet and work together as often as possible, either as the whole or a partial group. 
\paragraph{} As the project got a more steady pace, the next major issue was dealing with lack of experience in web development. Fortunately for the group, the assigned technical lead had lots of experience with this. Dealing with inexperience was done by ensuring every member understood the process of our development environment, installing all necessary tools and teaching every member the basics. 
\paragraph{} After this, all problems the group faced were identified and dealt with through meetings, during the sprint introductions and sprint retrospectives. In these steps we talked about what went well, what did not go so well and what should be done differently until next time. These things were typically individual, and were often solved by the respective members themselves, sometimes with help and sometimes without. 
\subsubsection{Internalizing the Scrum Process} During the beginning of the project we did not have a well defined work life cycle model, and our way of working were more similar to that of the Waterfall model than the Scrum model. As mentioned earlier, the group raised question to our ways of working as we had been doing in the beginning. Consequently, Scrum and Waterfall was thoroughly researched and presented to the group. After discussion, the life cycle model used by the group was changed from Waterfall to Scrum to be able to quickly respond to backlog changes and feedback from the customer. 
\paragraph{} Introducing Scrum also resulted in challenges for the team. Implementing the life cycle model in practice was not as easy as initially thought, and as a result work estimations had big deviations from actual work. In the period following the implementation of the Scrum life cycle model there were clear improvements of using the life cycle model, as well as in effiency and deviation, and reduction in assignment ambiguity. 
\subsubsection{Dealing with an Actively Changing Backlog} Weekly meetings were held with the customer to frequently ensure that the quality of the product was satisfactory and that the customer was satisfied with the progress and shape the product was taking. Because of this, the product backlog was constantly changing. Either by changing the prioritization of backlog items, introducing new items or deleting existing ones. To keep track of the items that made up the product, a complete backlog was created. Initially this was done as a spreadsheet, but we quickly noticed that a spreadsheet was not adaptable enough.
\paragraph{} As a solution to the backlog related issues, a project management tool called YouTrack was introduced and used by the group. YouTrack provided the necessary functionality to organize the project, and was customized to fit our needs. As YouTrack was a complex tool requiring too much work to constantly update to use as a sprint backlog tool, it ended being used as a product backlog tool only. As the sprint backlogs were not too extensive in the amount of issues, using Google Docs resulted in being better suited for our way of working. At the end of the project, the group went from being inexperienced in Scrum and project management, to being able to use the process to improve their way working.
\subsubsection{Risk Handling} During the initial phase of the project, the team developed a set of risks that could occur during the project. The risks are described in section \ref{sec:PlanningRiskMan}. To ensure the quality of the project and prevent risks, methods have been developed as listed in section \ref{sec:PlanningQuality}. Several of these risks, but fortunately not all, did occur. Using a proactive strategy we prevented the ones that could be prevented. The risk with the biggest impact on the project was the absence due to illness, travel, work, or similar. This could not be planned and resulted in a reduction of work hours. This was dealt with by working extra during some parts of the project, especially right before the pre-delivery and final delivery. 
\subsubsection{Customer} During our first meeting with the customer we were presented with the vision and ideas for the project, complemented by some sketches of how a prototype could look. Their project vision was a digital product that could give youth in urban areas of the world an alternative arena to express themselves, seemed vague with a lot of freedom. The customer placed their trust in the group, asking us to further develop their idea and to turn the project vision into a concrete product.
\paragraph{} Throughout the project, we have been in frequent contact with the customer. They have been enthusiastic about our work the entire time, asking questions, approving, and providing feedback on suggestions we have made. From the beginning of the project, the communication with the customer has been exceptional. Located in different areas, every meeting except for the first has been over Skype. For each customer meeting, our team has endeavored to deliver demos showing new features of the prototype. Using the share screen function Skype offers, we have been able to show the customer the current version of the product throughout the entire project. The customer has been satisfied with this form of communication.
\paragraph{} In October, the customer arranged a meeting in Oslo where the project was presented to the Youth Council of T{\o}yen area in Oslo. This provided an opportunity to get direct feedback from members of the main user group of the project, which later led to direct changes in the backlog and the creation of a mobile app. All in all, the customer has expressed that their satisfaction with the project, the prototype and working with us. The customer wishes to use the prototype and the report as a means of getting further funding for the project, and has offered the members of the group a place in the potential future of the system.
\subsubsection{Supervisor} The supervisor has been a great support throughout the entire project. Supervisor meetings have been held once a week, updating the supervisor on everything that happened in the past week. The supervisor's main focus has been to help the group with dynamics, targeting the difficulties and problems of the group and providing advice as to how to solve these. 

\newpage
\section{Conclusion} 
Looking at the entirety of the project, including the group dynamics, the product, the report, the communication with both the customer and the supervisor we are very satisfied with the results. Some of the greatest achievements lie in the progress of the group, for example the transition from an unstructured group to a cooperative unit capable of delivering a functional prototype as well as an extensive report describing the development process. The group has concluded that this is one of the most valuable learning experiences so far in our master studies, and the course most appliccable to real world situations and challenges.

\paragraph{} As a group we experienced the typical challenges and risks, dealing with them together, managing to get good cooperation and group dynamics despite personal and cultural differences. With every solved challenge, we have become more confident in our work, both individually and as a team. Not only did this project help us improve our group dynamics skills, but it also enchanced our ability to acquire and learn new skills. As most group members were inexperienced in both web and app development, this was a great oppurtunity to learn how to structure and create a web system from scratch. Continuous writing, editing and proofreading of the report led to improved language skills and learning how to properly phrase things on a professional level, which is a good preparation for writing a master thesis. Structuring and organizing every meeting, every work assignment, document and code line has taught us the importance of good documentation.
