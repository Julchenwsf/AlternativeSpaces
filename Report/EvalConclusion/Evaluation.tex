\chapter{Evaluation and Conclusion}
\label{chap:Eval}

\section{Evaluation} 
\paragraph{Evolution of the team} At the beginning of the course, the project teams of the course were randomly assigned. Each team member comprising the Alternative Spaces project development team come from different backgrounds, with different cultures, experiences and skills. Initially the most noticable challenge seemed to lie with language, with five norwegian and two international students. With the international students having too little experience with norwegian, the main communication of the group had to be in english. Fortunately, all group members are to a satisfying extent experienced with speaking english. 
\paragraph{Becoming a team} Upon meeting the customers and being introduced to a vague project without specific tasks of what was expected of us, we sat down as a group and began our initial phase of the project. This included regular group meetings where we sat down brainstorming the concept and idea development. This was among other things, deciding whether the prototype should be user oriented, interest oriented or location oriented. The project had a lot of freedom, but this also meant more responsibility for own work. Noticing our differences, the group dynamics seemed weak at this time. The group was unstructured and without a leader, and everything was done together as a group. 
\paragraph{Improving our work process} After the idea development had started converging towards an idea that the group and customers both seemed to like, the implementation and design phase started. As the group no longer worked together for each work session, the lack of group organization and difference in experience of each member in the group started having a negative effect on the work and workflow. This resulted in unclear tasks, which reduced the work efficiency of the group, lack in motivation and hours of work. To deal with this problem we sat down to identify the skills of each person, so that clear roles could be assigned each member. Among others, a project leader and a technical leader was assigned to organize and structure the group as well as keeping track on each member and being mainly responsible for dealing with problems. Additionally, work assignments now had to be more precisely defined so that when members were assigned a task, there was no ambiguity as to what the person should do, the time it should take and when it had to be done. As it turned out meeting every day was an impossible goal to have, these measures clearly made the entire team more effective and made it possible for every member to finish their work assignments within the given timeframes. Still, we tried to meet and work together as often as possible, either as the whole or a partial group. 
\paragraph{} As the project got a more steady pace, the next major issue was dealing with lack of experience in web development. Fortunately for the group, the assigned technical lead had lots of experience with this. Dealing with inexperience was done by ensuring every member understood the process of our development environment, installing all necessary tools and teaching every member the basic usage. After this, all problems the group faced were identified and dealt with through meetings, during the sprint introductions and sprint retrospectives. In these steps we talked about what went well, what didn’t go so well and what should be done differently until next time. These things were typically individual, and were often solved by the respective members themselves. Sometimes with help and sometimes without. 
\paragraph{Include the scrum process in our work} During the beginning of the project we didnt have a well defined work methodology, and our way of working was more similar to that of the waterfall model than the scrum model. As mentioned earlier, the group raised question to our ways of working as we had been doing in the beginning. Consequently, Scrum and Waterfall was thoroughly researched and presented to the group. Upon discussion, work methodology among the group was changed from waterfall to scrum to be able to quickly respond to backlog changes and feedback from the customer. Introducing Scrum also resulted in challenges for the team. Implementing the methodology in practice wasn’t as easy as initially thought, and as a result work estimations had big deviations from actual work. In the period following the implementation of the scrum methodology there were clear improvements of using the methodology, as well as in effiency and deviation, and reduction in assignment ambiguity. 
\paragraph{Dealing with an actively changing backlog} Weekly meetings were held with the customers to frequently ensure that the quality of the product was being kept and that the customers were satisfied with the progress and shape the product was taking. Because of this, our backlog was constantly changing. Either by changing the prioritization of backlog items, introducing new items or deleting existing ones. To keep track of the items that made up the product, a complete backlog was created. Initially this was done as a spreadsheet, but we quickly noticed that a spreadsheet wasn’t adaptable enough. Later, a project management tool called YouTrack was introduced and implemented to the group. YouTrack had more than enough functionality needed by the group, and was customized to fit our needs. As youtrack was a complex tool requiring too much work to constantly update to use as a sprint backlog tool, it ended being used as a product backlog tool only. As the sprint backlogs weren’t too extensive in the amount of issues, using google docs resulted in being better suited for our way of working. At the end of the project, the group went from being completely inexperienced in scrum and project management, to being able to use the process to improve their way working.
\paragraph{Rish Handling} During the initial phase of the project, the team developed a set of risks that could occur during the project. All the risks are described in the section \ref{sec:PlanningRiskMan}. To ensure the quality of the project and prevent risks, methods have been developed and listed in section \ref{sec:PlanningQuality}. Several of these risks, but fortunately not all of them, did occur. Using a proactive strategy we prevented the ones that could be prevented. The risk with the biggest impact on the project was the absence due to illness, travel, work or similar. This could not be planned and resulted in a reduction of work hours. This was dealt with by working extra during some parts of the project, especially right before the pre-delivery and final delivery. 
\paragraph{Customer} During our first meeting with the customers we were presented with the project vision, and their ideas complemented by some sketches of how a prototype could look like. Their project vision being a digital product that could give youths in urban areas of the world an alternative space to express themselves, seemed vague with a lot of freedom. The customers laid their trust in us, asking us to further develop an idea of how to turn this vision into a product. Throughout the project, we have been in frequent contact with the customers. They’ve been enthusiastic about out work the entire time, asking questions, approving and disapproving suggestions we’ve come with. From the beginning of the project, the communication with the customers has been exceptional. Located in different areas, every meeting except for the first has been over Skype. For each customer meeting, our team has aimed at delivering demos showing new features of the prototype. Using the share screen function Skype offers, we have been able to show the customers the current version of the product throughout the entire project. The customers have been satisfied with this form of communication. In October, the customers arranged a workshop in Oslo where the project was presented for the Youth Council in Oslo. This gave us the oppurtunity of getting direct feedback from people in our target group, which later led to direct changes in the backlog and the creation of a mobile app. All in all, the customers have expressed that they are satisfied with the project, the prototype and working with us, as we have with them. It has even been to the extent that the customers wish to use the prototype and the report as a means of getting further funding for the project, and have offered the members of the group a place in the potential future of the system.
\paragraph{Advisor} The advisor has been a great support throughout the entire project. Advisor meetings have been once a week, updating him on everything that’s happened over the past week. His main focuses has been at helping the group with dynamics, targeting difficulties and problems of the group and providing advice as to how to solve these. 

\newpage
\section{Conclusion} 

\paragraph{} Looking at the entirety of the project, including group dynamics, product, report, customers and advisor we’re extremely satisfied with the results. Some of the greatest achievements lie in the progress of the group, e.g. going from being an unstructured group to being a cooperative unit delivering a fully functional prototype and an extensive report describing the entire process. The group has concluded that this is without a doubt one of the most valuable learning experiences so far in our master studies, and the most relevant course yet taken to real-world challenges. 

\paragraph{} As a group we had to experience the typical challenges and risks, dealing with them together, managing to get good cooperation and group dynamics despite personal and cultural differences. With every solved challenge, we’ve become more confident in both our individial work and team work. Not only did this project help us improve our group dynamics skills, but also in learning new skills. As most group members were inexperienced in both web and app development, this was a great oppurtunity to learn how to structure and create a web system from scratch. Continuous writing, editing and proofreading of the report led to improving language skills and learning how to properly phrase things on a master level, which is a great preparation for the master thesis. Structuring and organizing every meeting, every work assignment, document and code line has taught us the importance of good documentation. 
