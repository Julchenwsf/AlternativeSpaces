\chapter{Project Reflection}
\label{chap:Eval}


\section{Introduction of the Project}

At the beginning of the course, the team was assigned Arbeidsforskningsinstituttet as the customer and the Alternative Spaces project. Upon meeting the customer, the group immediately started the concept development based on the vision provided by the customer. The group and the customer agreed that the vision was to be realised as a social network. Throughout the project, the entire group has gained extensive experience in different web and app development frameworks. Each member has learned how to develop and create interactive websites with complex functionality. 

\section{Evolution of the Team}

During the initial phase of the project, work was done together over regular meetings, five times a week. Progress was moderate, but inefficient, due to lack of structure and delegating responsibility between the team. As the concept converged towards an idea that both the group and the customer were satisfied with, the design and implementation phase started. 

\paragraph{} As the group started working more individually, the lack of group organization and difference in experience of each member in the group started having a negative effect on the work result and the workflow. To deal with this the group sat down identifying the skills of everyone, assigning clear roles to each member. The project leader and the technical leader stood responsible for structure both in developmental context and within the group. Additionally, introducing the Scrum process resulted in work being more precisely defined. There was no longer ambiguity as to what each person had to do, the time it should take and when it had to be done. These measures made the entire team more effective and made it possible for every member to finish their work assignments within the given timeframes, either working with the group or individually.

\section{Technical Problems}
As the project got to a more steady pace, the next major issue was dealing with lack of experience in web development. By teaching every member the basics of web development and installing all necessary tools for learning and developing individually, we ensured that every member felt comfortable in tackling their tasks. Despite this, as the project neared its end the development tasks became more complex, which finally resulted in the technical lead being constantly called for and overwhelmed with work. 

\paragraph{} Looking back we’ve been intrigued by the thought of doing some things differently. Introduction of Scrum and allocation of roles earlier in the process would bring structure and improve the work of the group at an earlier stage. Based on the feedback from the youth council in Oslo, realizing the entire product as a mobile application could have fitted better to the interests of modern youth.

\section{Customer}
From the beginning of the project, the customer placed great responsibility and trust in our group by giving us quite extensive freedom in taking the decisions. Spanning the entire project, communication with the customer has been exceptional. Weekly meetings were held with the customer to frequently ensure that the progress and shape the product, as well as the quality, was satisfactory. They were very enthusiastic about our work the entire time, asking questions, approving, and providing feedback on suggestions we have made. As a consequence, the product backlog was constantly changing. Either by changing the prioritization of backlog items, introducing new items or deleting existing ones. To keep track of the items that made up the product and deal with this problem, a complete backlog was created by using YouTrack. Despite being located in different areas, every meeting except for the first was over Skype. For each customer meeting, our team has managed to deliver demos showing new features of the prototype. 

\section{Supervisor} 
The supervisor has been a great support throughout the entire project. Supervisor meetings have been held once a week, updating the supervisor on everything that happened in the past week. Over the entire project, the supervisor has actively worked to help the with group dynamics, targeting difficulties and problems of the group, and providing advice as to how to solve these. 

Looking back we’ve been intrigued by the thought of doing some things differently. Introduction of Scrum and allocation of roles earlier in the process would bring structure and improve the work of the group at an earlier stage. Based on the feedback from the youth council in Oslo, realizing the entire product as a mobile application could have fitted better to the interests of modern youth.

Customer
From the beginning of the project, the customer placed great responsibility and trust in our group by giving us quite extensive freedom in taking the decisions. Spanning the entire project, communication with the customer has been exceptional. Weekly meetings were held with the customer to frequently ensure that the progress and shape the product, as well as the quality, was satisfactory. They were very enthusiastic about our work the entire time, asking questions, approving, and providing feedback on suggestions we have made. As a consequence, the product backlog was constantly changing. Either by changing the prioritization of backlog items, introducing new items or deleting existing ones. To keep track of the items that made up the product and deal with this problem, a complete backlog was created by using YouTrack. Despite being located in different areas, every meeting except for the first was over Skype. For each customer meeting, our team has managed to deliver demos showing new features of the prototype. 

Supervisor 
The supervisor has been a great support throughout the entire project. Supervisor meetings have been held once a week, updating the supervisor on everything that happened in the past week. Over the entire project, the supervisor has actively worked to help the with group dynamics, targeting difficulties and problems of the group, and providing advice as to how to solve these. 

\newpage
\chapter{Conclusion} 
Looking at the entirety of the project, including the group dynamics, the product, the report, the communication with both the customer and the supervisor we are very satisfied with the results. Some of the greatest achievements lie in the progress of the group, for example the transition from an unstructured group to a cooperative unit capable of delivering a functional prototype as well as an extensive report describing the development process. The group has concluded that this is one of the most valuable learning experiences so far in our master studies, and the course most applicable to real world situations and challenges.

\paragraph{} As a group we experienced the typical challenges and risks, dealing with them together, managing to get good cooperation and group dynamics despite personal and cultural differences. With every solved challenge, we have become more confident in our work, both individually and as a team. Not only did this project help us improve our group dynamics skills, but it also enhanced our ability to acquire and learn new skills. As most group members were inexperienced in both web and app development, this was a great opportunity to learn how to structure and create a web system from scratch. Continuous writing, editing and proofreading of the report led to improved language skills and learning how to properly phrase things on a professional level, which is a good preparation for writing a master thesis. Structuring and organizing every meeting, every work assignment, document and code line has taught us the importance of good documentation.
