\begin{abstract}
In an increasingly commercialized public space, youth have a diminished ability to use it freely. The research project "Alternative Spaces" aims to help decision makers (i.e. politicians, scientists, regular people) shape the public space to allow all citizens to freely enjoy it. This report details work done during the autumn of 2014 to develop a digital platform where youth citizens could express themselves freely about their interests and the space around them as part of the course TDT4290, Customer Driven Project, at NTNU. \\
The result is a webpage as well as a companion Android application to upload pictures to the webpage. It allows youth to interact with other youth around the world based on interests and locations, as well as comment on parts of the public space they would like changed. The system is to act as a prototype to display the ideas of the project, rather than the final system. \\
Chapter \ref{chap:Intro} introduces the project and explains the reasons behind it. Chapter 2 goes into the preliminary studies and planning. In chapter 3 and 4 we describe the design, system requirements, and overall architecture of our system. Chapter 5 describes our test plan and testing methods. Chapter 6 through 10 describes our five main sprints in detail. The final testing and project evaluation follows in chapter 11, while chapter 12 describes further areas of improvement. Finally, system documentation, risk review, and use cases can be found in the appendices.
\end{abstract}