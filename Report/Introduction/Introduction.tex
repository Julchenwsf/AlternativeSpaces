\chapter{Introduction}

\paragraph{} This chapter will go through the basics of our project. This includes the people involved as well as its background and purpose.

\section{Project Name} 
This is a report on the project "Alternative Spaces - the digital prototype", herein referred to as "ASDP", or "The Digital Prototype"\\
\indent It is part of the larger research project "Alternative Spaces", which explores the possibilities of creating alternative city spaces in cooperation with urban youth, artists, architects, journalists and scientists. It is currently planning to run a test project in T\o yen in Oslo.

\section{Background and Description}
This project is part of the course TDT4290 at NTNU. The purpose of the course is to give students experience working on IT projects, by making the projects as close to a real working situation as possible. Steps involved in the project includes, but is not limited to, planning and management, software development, customer relations, and group dynamics.
\paragraph*{} This group was given the task of developing a platform to facilitate "digital sharing of [visual, physical and mediated expressions] for youth." The aim is mainly to give ideas of how such a platform could work, and to have a prototype ready for a test in the T\o yen area of Oslo.
\paragraph*{} The description of the project from AFI is as follows:
\paragraph*{} \textit{With the increasing commercialization of city spaces, youth citizens have less room to express themselves, whether it is through visual, physical or mediated expressions like movement, dance, street art, sports activities or media outlets. The research project “Alternative Spaces” will explore the possibilities of creating alternative city spaces (both material and virtual) together with groups of youth, dancers, artists, landscape architects, journalists and scientists. Visual expression, digital media and film will make up both the methodological and empirical part of the project. \\
\indent This assignment is to design a new platform experience that can facilitate digital sharing of such expressions across urban communities globally. As many of the project activities will be happening “on the ground” there is an acute need for a commercial-free digital hub where youth can upload, share and create content that strengthens their local engagement, expression of ideas and experimental thinking when it comes to participation and political influence on the development of their city spaces.\\
\indent The project is in the early phase of development, where idea generation and the creating of connections and network of researchers and practitioners are integral. There will thus be plenty of opportunities for the students to influence the overall project design. During the fall of 2014 the project will arrange a workshop where all relevant actors will be invited to work together on related issues. This will be a venue for the students in the “customer driven project” to engage directly with the potential users of their platform. The case study of the first phase of the project will be Oslo, but Trondheim may be included as a complementary case if the student group finds the project attractive. \\
The project assignment will consist of the following activities:
\begin{enumerate}
\item Produce a user-oriented case, in dialogue with the research group
\item Participate and present pilot idea in “Alternative Spaces” workshop
\item Produce design suggestions, map needs and find the relevant technology for the pilot
\item Develop the web interface for analysis and representation
\item Evaluate design and technical solution and recommend further development
\end{enumerate}}
\paragraph*{} We decided to create a web page focused on the idea of planning open events and sharing media, based around the idea of interests and location. It allows you to search for images on a map based on these interests, and will show matches. To more easily facilitate uploading of pictures, we also created an accompanying Android application for this purpose. This app currently only has that functionality with no more planned from us. A thorough description of ideas for further development can be found in chapter 7

\section{Customer}
Our customer is the Work Research Institute (AFI), represented by the researchers in charge of the "Alternative Spaces" project, Aina Landsverk Hagen and Arne Bygd\aa s.
\subparagraph{}The Work Research Institute (AFI) has as its main goal to gain knowledge on work environment organization and leadership, based in research. Of particular interest are organizational and leadership structures that strengthen conditions for learning, business development, involvement and restructuring in both the public and private sector.\\
\begin{figure}[ht!]
\centering
\includegraphics[width=90mm]{./Introduction/img/afi.png}
\caption{AFI \label{overflow}}
\end{figure}

\section{Group Organization}
\paragraph{} Our group consists of seven people, and most of them have more than one role within the team. There are three main roles in a scrum team; product owner, scrum master and scrum team. The overall structure of this group is flat, since the group is fairly small. 

\subsection{Roles}
\paragraph*{} \textbf{The Project Lead/Scrum Master} is responsible for the overall progress of the project and organizational tasks such as planning and workload. The project lead has to make plans for sprints, meetings and deliveries and maintain contact with the customer and supervisor.

\paragraph*{} \textbf{Team}
\begin{itemize}
\item \textbf{The Implementation Lead} has final responsibility over the system. He is responsible for keeping track of what needs to be done on the system to make it ready for delivery. He also has a responsibility to be available to help other with any problems they might have regarding the implementation.
\item \textbf{The Requirements Specialist's} initial responsibility is making sure a system requirements document is created and important use cases are ready. They also have a responsibility to make sure the final system fulfills these requirements.
\item \textbf{The Document Manager} is responsible for the work done on this document. They are responsible for assigning sections that need to be written in the document and making sure the different parts match each other.
\item \textbf{The Test Manager} works closely with the requirements specialist to make tests for the final system test, as well as doing tests while development is ongoing. They are responsible for making a test plan and making sure the plan is executed in the final stages of the project.
\item \textbf{Developers} are responsible for creating quality code for any task they may be assigned to, and keeing the repository up to date with their code. They also have to make sure to keep the time spent up to date on our backlog, and to take responsibility for any open issues they are able to solve.
\item \textbf{Architects and designers} are responsible for the overarching design and higher level system architecture. As with developers, the have to make sure to keep time spent up to date, and to be ready to take care of any issues.
\end{itemize}

\subsection{Allocation of Roles}
Each member of our team has at least one role with its accompanying responsibilities. We selected these based on the needs of the project and the wishes of the group. In addition to any roles allocated here, all group members are expected to be able to jump in and do a critical task as needed, and the Project Lead is responsible for keeping track and allocating these tasks. We tried to keep the number of roles down, since the group is fairly small.

\begin{itemize}
\item \textbf{Project Lead: } Jonas F. Therkelsen
\item \textbf{Implementation Lead: } Valerij Fredriksen
\item \textbf{Requirements Specialist: } Brage E. Jahren
\item \textbf{Document Manager: } Julia Schneider
\item \textbf{Test Manager: } Manasa Vallamkondu
\item \textbf{Developers: } Yngve Bloch-Hoell, Valerij Fredriksen, Brage E. Jahren, Hans Olav Slotte, Manasa Vallamkondu
\item \textbf{Architects and designers: } Valerij Fredriksen, Brage E. Jahren, Jonas F. Therkelsen 
\end{itemize}

%\chapter{Project Plan}

\section{Overall Project Plan}
\begin{itemize}
\item \textbf{Project Name:} Alternative Spaces
\item \textbf{Project Sponsor:} Work Research Institute (AFI), Akershus University College of Applied Sciences
\item \textbf{Partners:}
\item \textbf{Background:} The increasing commercialization of city spaces means youth citizens have less room to express themselves. There is need to explore the possibility of new alternative spaces.
\item \textbf{Measurement of project effects:} TODO
\item 
\item \textbf{Available person-hours:} 1900
\item \textbf{Schedule: } 
  \begin{itemize}
  \item \textbf{12 Sep.} System requirements. 
  \item \textbf{17 Oct.} Final report table of contents
  \item \textbf{20 Nov.} Final Presentation
  \end{itemize}
\end{itemize}


\section{Concrete Project Plan}
\subsection{Phases}
\subsection{Activities}
\subsection{Milestones}
\begin{tabular}{|l|r|r|}
\hline
\textbf{Phase}         	  & \textbf{\%} & \textbf{Hours} \\
\hline
Project Management  	  & 		 10 & 		     190 \\
Lectures and self study   & 		 10 & 			 190 \\
Planning				  &  		  7 & 			 133 \\
Pre study				  & 		 15 & 			 285 \\
Requirement spec		  & 		 20 & 			 380 \\
Design					  & 		 15 & 			 285 \\
Programming/documentation & 		 13 & 			 247 \\
Project evaluation		  &  		  5 &  			  95 \\
Presentation and demo	  &  		  5 &  			  95 \\
\hline
\textbf{Total}			  &			100 &			1900 \\
\hline
\end{tabular}

\section{Project Organization}
\subsection{Organizational diagram}
\subsection{Roles}
\begin{itemize}
\item \textbf{Project Lead:} Jonas F. Therkelsen
\item \textbf{Vice Project Lead:} Julia Schneider
\item \textbf{Implementation:} 
  \begin{itemize}
  \item \textbf{Lead:} Valerij Fredriksen
  \item Yngve S. Bloch-Hoell
  \item Brage E. Jahren
  \item Julia Schneider  
  \item Hans Olav Slotte
  \item Jonas F. Therkelsen
  \item Manasa Vallamkondu
  \end{itemize}   
\item \textbf{Design:}
\end{itemize}
\subsection{Responsibilities}
\subsection{Weekly Schedule}

\section{Templates and Standard}

\section{Version Control}
We will be using GitHub for version control in this project, both for the code and the Report.

\section{Documentation of project work}

\section{Quality Assurance}

\section{Test plan}

\section{Duration and work-hours} This project lasted for 12 weeks (25/8 - 2014 to 20/11 - 2014). We estimated a total workload of 1900 work-hours.