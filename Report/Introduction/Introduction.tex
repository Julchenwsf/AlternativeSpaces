\chapter{Introduction}
\label{chap:Intro}
This chapter presents the fundamentals of our project. This includes the background and purpose of it as well as the people involved.

\section{Project Name}
\label{sec:IntroProjName}
This is a report on the project "Alternative Spaces - the digital prototype", herein referred to as "ASpaces", or "The Digital Prototype". \\
It is part of the larger research project "Alternative Spaces", which explores the possibilities of creating alternative city spaces in cooperation with urban youth, artists, architects, journalists and scientists. It is currently planning to run a test project in Tøyen in Oslo.

\section{Project Description}
\label{sec:IntroProjectDescr}
This project is part of the course TDT4290: Customer Driven Project at NTNU. The purpose of the course is to give students experience working on IT projects, by making the projects as close to a real working situation as possible. Steps involved in the project includes, but are not limited to, planning and management, software development, customer relations, and group dynamics. \\
This group was given the task of developing a platform to facilitate "digital sharing of [visual, physical and mediated expressions] for youth." The aim is mainly to give ideas of how such a platform could work, and to create a prototype capable of demonstrating what a complete system would be capable of. \\
The description of the project from AFI is as follows:

\textit{"With the increasing commercialization of city spaces, youth citizens have less room to express themselves, whether it is through visual, physical or mediated expressions like movement, dance, street art, sports activities or media outlets. The research project “Alternative Spaces” will explore the possibilities of creating alternative city spaces (both material and virtual) together with groups of youth, dancers, artists, landscape architects, journalists and scientists. Visual expression, digital media and film will make up both the methodological and empirical part of the project. \\
This assignment is to design a new platform experience that can facilitate digital sharing of such expressions across urban communities globally. As many of the project activities will be happening “on the ground” there is an acute need for a commercial-free digital hub where youth can upload, share and create content that strengthens their local engagement, expression of ideas and experimental thinking when it comes to participation and political influence on the development of their city spaces. \\
The project is in the early phase of development, where idea generation and the creating of connections and network of researchers and practitioners are integral. There will thus be plenty of opportunities for the students to influence the overall project design. During the fall of 2014 the project will arrange a workshop where all relevant actors will be invited to work together on related issues. This will be a venue for the students in the “customer driven project” to engage directly with the potential users of their platform. The case study of the first phase of the project will be Oslo, but Trondheim may be included as a complementary case if the student group finds the project attractive. 
The project assignment will consist of the following activities:
\begin{itemize}
\item Produce a user-oriented case, in dialogue with the research group
\item Participate and present pilot idea in “Alternative Spaces” workshop
\item Produce design suggestions, map needs and find the relevant technology for the pilot
\item Develop the web interface for analysis and representation
\item Evaluate design and technical solution and recommend further development"
\end{itemize}}

We decided to create a web application focused on the idea of connecting urban youth sharing similar interests or hobbies by planning open events and sharing media at different locations. It allows you to search for media and events on a map based on these interests, and will show the most relevant matches. To more easily facilitate the uploading of pictures, we have also created an accompanying Android application for this purpose. This app currently only has the functionality of sharing media, to better illustrate how users would interact with the system for the customer. A thorough description of ideas for further development can be found in chapter INSERT REFERENCE HERE (EXAMPLE IN COMMENT) %\ref{chap:FurtherWork}.


\section{Customer}
\label{sec:IntroCustomer}

Our customer is the Work Research Institute (AFI), represented by the researchers in charge of the "Alternative Spaces" project, Aina Landsverk Hagen and Arne Bygdås. \\
The main goals of the institute is to produce research based knowledge on organization, leadership, and the work environment. Of particular interest are organizational and leadership structures that strengthen conditions for learning, business development, involvement and restructuring in both the public and private sector.

\begin{figure}[ht!]
\centering
\includegraphics[width=\linewidth]{./Introduction/img/afi}
\caption{The Work Research Institute \label{fig:IntroAfi}}
\end{figure}

\section{Stakeholders}
\label{sec:IntroStakeholders}

The stakeholders of this project are the people directly or indirectly involved in the development or results of the project. The stakeholders have an interest in the project and are either affected by or affect the development process or the end results. Four stakeholders of the Alternative Spaces project have been identified.

\subsection{Project Team}
\label{subsec:IntroStakeProjectTeam}

The project team is composed of seven students from the Norwegian University of Technology (NTNU). All members of the team are on their Master's degree either in Informatics or Computer Science Engineering. It's an international team with five students native to the University and two exchange students, one from India and one from Germany. The diversity adds cultural flavor and variety of viewpoint. The project team, counselled by the advisor, is mainly focused on satisfying the customer by delivering a fully functional prototype of the product within the given timeframe. To achieve this the team works with the system development, testing, and documentation of the Alternative Spaces project.

\begin{minipage}{\linewidth}
\centering
%\setlength{\tabcolsep}{22pt}
\textbf{Team:} \\
\smallskip
\rowcolors{1}{blue!20}{blue!10}
\begin{tabular}{ |p{5cm} p{35mm}| }
	\hline
	\cellcolor{gray!25} \begin{tabular}{l} \cellcolor{gray!25} \textbf{Name and E-Mail} \end{tabular} & \cellcolor{gray!25} \textbf{Study} \\
	\hline
	\begin{tabular}{l} Jonas Foyn Therkelsen \\ \texttt{jonasft@stud.ntnu.no} \end{tabular} & Computer Science, Software \\
	\begin{tabular}{l} Brage Ekroll Jahren \\ \texttt{brageej@stud.ntnu.no} \end{tabular} & Computer Science, Artificial Intelligence \\
	\begin{tabular}{l} Valerij Fredriksen \\ \texttt{valerijf@stud.ntnu.no} \end{tabular} & Computer Science, Artificial Intelligence \\
	\begin{tabular}{l} Hans Olav Slotte \\ \texttt{slotte@stud.ntnu.no} \end{tabular} & Computer Science, Artificial Intelligence \\
	\begin{tabular}{l} Yngve S. Bloch-Hoell \\ \texttt{yngvesbl@stud.ntnu.no} \end{tabular} & Computer Science, Artificial Intelligence \\
	\begin{tabular}{l} Julia Schneider \\ \texttt{juliasch@stud.ntnu.no} \end{tabular} & Informatics \\
	\begin{tabular}{l} Manasa Vallamkondu \\ \texttt{manasav@stud.ntnu.no} \end{tabular} & Informatics \\
	\hline
\end{tabular}
\end{minipage}

\subsection{Customer}
\label{subsec:IntroStakeCustomer}
The customer, also specified in the section above, is the Work Research Institute (AFI), represented by the researchers in charge of the Alternative Spaces project, Aina Landsverk Hagen and Arne Bygdås. The customers are concerned with the iterative deliverables from each sprint, evaluating them and giving feedback to the Scrum Team so that the product will meet their expectations.

\begin{minipage}{\linewidth}
\centering
\setlength{\tabcolsep}{22pt}
\textbf{Customer:} \\
\smallskip
\rowcolors{1}{blue!20}{blue!10}
\begin{tabular}{|l l|}
    \hline
    \cellcolor{gray!25} \textbf{Name} & \cellcolor{gray!25} \textbf{E-mail} \\
    \hline
    Aina Landsverk Hagen & \texttt{aina.hagen@afi.hioa.no} \\
    Arne Bygd\aa s & \texttt{arne.bygdas@afi.hioa.no} \\
    \hline
\end{tabular}
\end{minipage}

\subsection{Advisor}
\label{subsec:IntroStakeAdvisor}

The advisor of the group is Gleb Sizov of IDI at NTNU. As the advisor of the Alternative Spaces group, he is concerned with the progression of the group, focusing on group dynamics and the progress of system development and the report. 

\begin{minipage}{\linewidth}
\centering
\setlength{\tabcolsep}{22pt}
\textbf{Advisor:} \\
\smallskip
\rowcolors{1}{blue!20}{blue!10}
\begin{tabular}{|l l|}
    \hline
    \cellcolor{gray!25} \textbf{Name} & \cellcolor{gray!25} \textbf{E-mail} \\
    \hline
    Gleb Sizov & \texttt{glebsv@gmail.com} \\
    \hline
\end{tabular}
\end{minipage}

\subsection{End Users}
\label{subsec:IntroStakeEndUsers}
These stakeholders are the end users of the complete system that will be Alternative Spaces. Currently there are no users, but the targeted audience is youth living in urban regions with high population densities. The end users want to use the site to satisfy social, entertainment and educational needs. Currently only a few external users have tested and evaluated the product and given valuable feedback that directly influenced the design of the prototype.