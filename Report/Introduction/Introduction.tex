\chapter{Introduction}

\paragraph{} This chapter will go through the basics of our project. This includes the people involved as well as its background and purpose.

\section{Project Name} 
This is a report on the project "Alternative Spaces - the digital prototype", herein referred to as "ASDP", or "The Digital Prototype"\\
\indent It is part of the larger research project "Alternative Spaces", which explores the possibilities of creating alternative city spaces in cooperation with urban youth, artists, architects, journalists and scientists. It is currently planning to run a test project in T\o yen in Oslo.

\section{Background and Description}
This project is part of the course TDT4290 at NTNU. The purpose of the course is to give students experience working on IT projects, by making the projects as close to a real working situation as possible. Steps involved in the project includes, but is not limited to, planning and management, software development, customer relations, and group dynamics.
\paragraph*{} This group was given the task of developing a platform to facilitate "digital sharing of [visual, physical and mediated expressions] for youth. The aim is mainly to give ideas of how such a platform could work, and to have a prototype ready for a test in the T\o yen area of Oslo.
\paragraph*{} The description of the project from AFI is as follows:
\paragraph*{} \textit{With the increasing commercialization of city spaces, youth citizens have less room to express themselves, whether it is through visual, physical or mediated expressions like movement, dance, street art, sports activities or media outlets. The research project “Alternative Spaces” will explore the possibilities of creating alternative city spaces (both material and virtual) together with groups of youth, dancers, artists, landscape architects, journalists and scientists. Visual expression, digital media and film will make up both the methodological and empirical part of the project. \\
\indent This assignment is to design a new platform experience that can facilitate digital sharing of such expressions across urban communities globally. As many of the project activities will be happening “on the ground” there is an acute need for a commercial-free digital hub where youth can upload, share and create content that strengthens their local engagement, expression of ideas and experimental thinking when it comes to participation and political influence on the development of their city spaces.\\
\indent The project is in the early phase of development, where idea generation and the creating of connections and network of researchers and practitioners are integral. There will thus be plenty of opportunities for the students to influence the overall project design. During the fall of 2014 the project will arrange a workshop where all relevant actors will be invited to work together on related issues. This will be a venue for the students in the “customer driven project” to engage directly with the potential users of their platform. The case study of the first phase of the project will be Oslo, but Trondheim may be included as a complementary case if the student group finds the project attractive. \\
The project assignment will consist of the following activities:
\begin{enumerate}
\item Produce a user-oriented case, in dialogue with the research group
\item Participate and present pilot idea in “Alternative Spaces” workshop
\item Produce design suggestions, map needs and find the relevant technology for the pilot
\item Develop the web interface for analysis and representation
\item Evaluate design and technical solution and recommend further development
\end{enumerate}}

\section{Customer}
Our customer is the Work Research Institute (AFI), represented by Aina Landsverk Hagen and Arne Bygd\aa s.
\subparagraph{}The Work Research Institute (AFI) has as its main goal to gain knowledge on work environment organization and leadership, based in research. Of particular interest are organizational and leadership structures that strengthen conditions for learning, business development, involvement and restructuring in both the public and private sector.\\
\begin{figure}[ht!]
\centering
\includegraphics[width=90mm]{./Introduction/img/afi.png}
\caption{AFI \label{overflow}}
\end{figure}

\section{Team}
\begin{itemize}
\item \textbf{Jonas F. Therkelsen:} Project Lead and Customer Contact
\item \textbf{Julia Schneider:} Vice Lead
\item \textbf{Valerij Fredriksen:} Implementation Lead
\end{itemize}

\section{Stakeholders}

The stakeholders of this project are the people directly or indirectly involved. The stakeholders have an interest in the project and are either affected or affect the development process or the end results. The Alternative Spaces project have four stakeholders. 

\p{\textbf{Project Team}} 
The project team is composed of seven students from the Norwegian University of Technology (NTNU). All the members of the team are on their Master's degree either in Informatics or Computer Science Engineering. It's an international team with five students native to the University and two exchange students. One from India and one from Germany. The diversity adds culturl flavor and variety of viewpoint. The project team, counselled by the advisor, is mainly focused on satisfying the customer by delivering a fully functional prototype of the product within a given timeframe. To achieve this the team works with the system development, testing and the documentation of the Alternative Spaces project. 
\p{\textbf{Customer}} 
The customer are two researchers, Aina Landsverk Hagen and Arne Bygdaas of Arbeidsforskningsinstituttet (AFI) located in Oslo. 
\p{\textbf{Advisor}} 
\p{\textbf{End Users}} 

\section{Duration and work-hours} This project lasted for 12 weeks (25/8 - 2014 to 20/11 - 2014). We estimated a total workload of 1900 work-hours.