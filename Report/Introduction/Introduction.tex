\chapter{Introduction}

\paragraph{} This chapter will go through the basics of our project. This includes the people involved as well as its background and purpose.

\section{Project Name} 
This is a report on the project "Alternative Spaces - the digital prototype", herein referred to as "ASDP", or "The Digital Prototype\\
\indent It is part of the larger research project "Alternative Spaces", which explores the possibilities of creating alternative city spaces in cooperation with urban youth, artists, architects, journalists and scientists. It is currently planning to run a test project in T\o yen in Oslo.

\section{Customer}
The Work Research Institute (AFI) has as its main goal to gain knowledge on work environment organization and leadership, based in research. Of particular interest are organizational and leadership structures that strengthen conditions for learning, business development, involvement and restructuring in both the public and private sector.\\
\indent Our particular customers are the researchers in charge of the "Alternative Spaces" project, Aina Landsverk Hagen and Arne Bygd\aa s.

\section{Team}
\begin{itemize}
\item \textbf{Jonas F. Therkelsen:} Project Lead and Customer Contact
\item \textbf{Julia Schneider:} Vice Lead
\item \textbf{Valerij Fredriksen:} Implementation Lead
\end{itemize}

\section{Background and Description}
With the increasing commercialization of city spaces, youth citizens have less room to express themselves, whether it is through visual, physical or mediated expressions like movement, dance, street art, sports activities or media outlets. The research project “Alternative Spaces” will explore the possibilities of creating alternative city spaces (both material and virtual) together with groups of youth, dancers, artists, landscape architects, journalists and scientists. Visual expression, digital media and film will make up both the methodological and empirical part of the project. \\
\indent This assignment is to design a new platform experience that can facilitate digital sharing of such expressions across urban communities globally. As many of the project activities will be happening “on the ground” there is an acute need for a commercial-free digital hub where youth can upload, share and create content that strengthens their local engagement, expression of ideas and experimental thinking when it comes to participation and political influence on the development of their city spaces.\\
\indent The project is in the early phase of development, where idea generation and the creating of connections and network of researchers and practitioners are integral. There will thus be plenty of opportunities for the students to influence the overall project design. During the fall of 2014 the project will arrange a workshop where all relevant actors will be invited to work together on related issues. This will be a venue for the students in the “customer driven project” to engage directly with the potential users of their platform. The case study of the first phase of the project will be Oslo, but Trondheim may be included as a complementary case if the student group finds the project attractive. \\
The project assignment will consist of the following activities:
\begin{enumerate}
\item Produce a user-oriented case, in dialogue with the research group
\item Participate and present pilot idea in “Alternative Spaces” workshop
\item Produce design suggestions, map needs and find the relevant technology for the pilot
\item Develop the web interface for analysis and representation
\item Evaluate design and technical solution and recommend further development
\end{enumerate}

%\chapter{Project Plan}

\section{Overall Project Plan}
\begin{itemize}
\item \textbf{Project Name:} Alternative Spaces
\item \textbf{Project Sponsor:} Work Research Institute (AFI), Akershus University College of Applied Sciences
\item \textbf{Partners:}
\item \textbf{Background:} The increasing commercialization of city spaces means youth citizens have less room to express themselves. There is need to explore the possibility of new alternative spaces.
\item \textbf{Measurement of project effects:} TODO
\item 
\item \textbf{Available person-hours:} 1900
\item \textbf{Schedule: } 
  \begin{itemize}
  \item \textbf{12 Sep.} System requirements. 
  \item \textbf{17 Oct.} Final report table of contents
  \item \textbf{20 Nov.} Final Presentation
  \end{itemize}
\end{itemize}


\section{Concrete Project Plan}
\subsection{Phases}
\subsection{Activities}
\subsection{Milestones}
\begin{tabular}{|l|r|r|}
\hline
\textbf{Phase}         	  & \textbf{\%} & \textbf{Hours} \\
\hline
Project Management  	  & 		 10 & 		     190 \\
Lectures and self study   & 		 10 & 			 190 \\
Planning				  &  		  7 & 			 133 \\
Pre study				  & 		 15 & 			 285 \\
Requirement spec		  & 		 20 & 			 380 \\
Design					  & 		 15 & 			 285 \\
Programming/documentation & 		 13 & 			 247 \\
Project evaluation		  &  		  5 &  			  95 \\
Presentation and demo	  &  		  5 &  			  95 \\
\hline
\textbf{Total}			  &			100 &			1900 \\
\hline
\end{tabular}

\section{Project Organization}
\subsection{Organizational diagram}
\subsection{Roles}
\begin{itemize}
\item \textbf{Project Lead:} Jonas F. Therkelsen
\item \textbf{Vice Project Lead:} Julia Schneider
\item \textbf{Implementation:} 
  \begin{itemize}
  \item \textbf{Lead:} Valerij Fredriksen
  \item Yngve S. Bloch-Hoell
  \item Brage E. Jahren
  \item Julia Schneider  
  \item Hans Olav Slotte
  \item Jonas F. Therkelsen
  \item Manasa Vallamkondu
  \end{itemize}   
\item \textbf{Design:}
\end{itemize}
\subsection{Responsibilities}
\subsection{Weekly Schedule}

\section{Templates and Standard}

\section{Version Control}
We will be using GitHub for version control in this project, both for the code and the Report.

\section{Documentation of project work}

\section{Quality Assurance}

\section{Test plan}

\section{Duration and work-hours} This project lasted for 12 weeks (25/8 - 2014 to 20/11 - 2014). We estimated a total workload of 1900 work-hours.