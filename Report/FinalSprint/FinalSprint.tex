\chapter{Final Phase}
\label{chap:FinalPhase}

The product prototype was finally finished during Sprint 5. A fully functional website and mobile app containing most of the functionalities previously specified. In this final phase of the project, the main goal has been to finish the project report, test the prototypes and fix any bugs that were found.

\section{Sprint Plan}
\label{sec:FinalPlan}

During the final phase the product prototypes mainly went through testing and bug fixing. The main focus of this phase was, however, completing the report of the project and preparation of the presentation. This final sprint started with each person uploading their work to the shared drive, followed by each member proof reading, editing and correcting the work before the final version was implemented in latex. Other tools, like the work sheets and youtrack, had their finishing updates.

\paragraph{} The responsible people for the report did the main proof reading, editing and correction. Each person responsible for sections that were lacking, had to finish their sections and upload their work so that the group could finish the report. Another focus was to prepare for the presentation, to be held November 20th, 2014. Main tasks of preparation included creating slides and rehearse presenting.

\paragraph{} The development team was reduced to only one person who was responsible for setting up the website to www.mysplot.com and the app to Google Play, bug fixing and final touches if necessary. Each member of the team had to execute a set of test cases and report the results to ensure that the product was ready for delivery. The test cases can be found in the appendix.
The tasks to be finished are mentioned in detail in the Sprint backlog.

\section{Sprint Workload and Duration}
\label{sec:FinalWorkload}
%
\begin{minipage}{\linewidth}
\centering
\setlength{\tabcolsep}{22pt}
\textbf{Sprint 2:} 
\smallskip
\rowcolors{1}{blue!20}{blue!10}
\begin{tabular}{ |l l| }
	\hline
	\it{Duration} & 2 weeks \\
	\it{Start} & November 3rd. \\
	\it{End} & November 16th. \\
	\it{Workload} & Hours spent by the entire group on Sprint 2. \\
	\it{Goal} & 20-25 hours per person \\
	\hline
\end{tabular}
\end{minipage}
\bigskip
%
\begin{minipage}{\linewidth}
\setlength{\tabcolsep}{25pt}
\centering
\rowcolors{1}{blue!20}{blue!10}
\begin{tabular}{ |l|l| }
	\hline
	\multicolumn{2}{|c|}{\cellcolor{gray!25} Workload} \\
	\hline
	\it{Planning} & 60 hrs \\
	\it{Development} & 5 hrs \\
	\it{Design} & None \\
	\it{Documentation} & 237 hrs \\
	\it{Testing \& Bug Fixing} & 19 hrs \\
	\hline
	\it{Total workload} & 327 hrs \\
	\hline
\end{tabular}
\captionof{table}{\label{tab:FinalWorkload} Workload of Final Phase} 
\end{minipage}

\section{Sprint Backlog}
\label{sec:FinalBacklog}

Below is a table depicting the sprint backlog for the final phase. The estimated and actual time are also displayed.

\begin{minipage}{\linewidth}
\setlength{\tabcolsep}{12pt}
\centering
\rowcolors{1}{blue!20}{blue!10}
\begin{tabular}{|p{1cm}|p{4cm}|p{2cm}|p{2cm}|}
\hline
\cellcolor{gray!25} ID & \cellcolor{gray!25} Description & \cellcolor{gray!25} Estimated Time & \cellcolor{gray!25} Actual Time \\
\hline
RPT-8 & \it{Further Work completion} & 30 hours & 41 hours\\
RPT-15 & \it{Appendices completion} & 15 hours & ? \\
RPT-18 & \it{Test Cases and Test Results} &  2 hours & 4 hours \\
RPT-19 & \it{Testing and Documentation} & 15 hours & ? \\
RPT-42 & \it{Sprint 3 completion} & 15 hours & ? \\
RPT-43 & \it{Sprint 4 completion} & 15 hours & ? \\
RPT-44 & \it{Final Phase completion} & 15 hours & ? \\
RPT-76 & \it{Architecture completion} & 15 hours & ? \\
RPT-78 & \it{Latex Implementation} &  & ? \\
RPT-79 & \it{Slide presentation} & 15 hours & ? \\
\hline
\end{tabular}
\captionof{table}{Backlog for sprint 1.} 
\end{minipage}

\subsection{Completing the report}
\label{sec:FinalReport}

To keep track of which chapters for the report are incomplete the team uses a customized google docs including all chapters which are marked differently based on status. The colors presented the status of this chapter which are:

\begin{figure}[ht!]
\centering
\includegraphics[width=90mm]{./FinalSprint/img/Colors.png}
\caption{Colors used in for the Sprint report backlog. \label{fig:FinalColors}}
\end{figure}

\begin{figure}[ht!]
\centering
\includegraphics[width=90mm]{./FinalSprint/img/ReportBacklog.png}
\caption{A remade Sprint Report Backlog, made for quickly keeping status of the report development. \label{fig:FinalReportBacklog}}
\end{figure}

\section{Testing}
\label{sec:FinalTesting}

\section{Results}
\label{sec:FinalResults}

The product prototype was basically finished in the last sprint. Each member tested the final version of the product from the user perspective, reporting bugs and issues which were actively corrected as they were reported. The bug searching and fixing ensured us a stable and fully functional version of the website and app ready for delivery. 

\paragraph{} For the project reflection, each member had to supply their experiences with the entire project. With this reflection, the group learned about the challenges and progress the members of the group had faced and achieved individually and together. 

\paragraph{} The major result of the sprint was finishing the report. During the sprint each member involved in the report had to ensure that every section they were responsible for was completed. All parts finished are specified further in the Sprint Plan. Proof reading, editing, correcting of grammar and spelling errors, all contributed to improving every section of the report. Finally, the report was implemented in latex giving it a professional touch. This was time consuming, but gave the report better consistency in the end. 

\paragraph{} The concluding tasks of the sprint was preparation for the presentation to be held for both the customers and the examiner. This is an ongoing task. Hopefully the slides, demo, and presentation will go as planned and well. 


\section{Sprint Retrospective}
\label{sec:FinalRetrospective}

\subsection{Start Doing}
\begin{itemize}
\item Project Reflection
\item Prepare presentation
\item Print report
\end{itemize}

\subsection{Stop doing}
\begin{itemize}
\item Taking too long breaks during work
\item Inefficient meddling in documents
\item Sleeping way into the day
\item Being ineffective in the mornings
\end{itemize}

\subsection{Continue Doing}
\begin{itemize}
\item Working effectively when in the flow! 
\item Proof reading
\item Latex implementation
\item Taking awesome notes during meetings! 
\end{itemize}

\section{Group Dynamics}
\label{sec:FinalDynamics}
Over the entire project the team has made huge progress in working together. This is evident in the way the working process has changed and improved as each sprint has come to pass. Every member has given their best and spent as much time on the project as they could manage. Upon finishing this last sprint every member seems proud of the product, the report, and all their work efforts realised over the project. As well as the progress and experience they made in working as a team and group dynamics, and acquiring new knowledge and skills. In this last sprint the entire group has greatly contributed in finishing a fully functional product and an extensive report describing every aspect of the project, in a smooth work flow that is incomparable to our first weeks of working together. 